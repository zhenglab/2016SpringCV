\documentclass[notheorems,serif,table,compress]{beamer}  %dvipdfm选项是关键,否则编译统统通不过
%%------------------------常用宏包------------------------
%%注意, beamer 会默认使用下列宏包: amsthm, graphicx, hyperref, color, xcolor, 等等
\usepackage{fontspec,xunicode,xltxtra}  % for XeTeX
\usepackage{verbatim}
\usepackage{mathabx}
\usepackage{latexsym}
\usepackage{amsfonts,amssymb}
\usepackage{styles/iplouclistings}
\usepackage{fancybox}
\usepackage{colortbl}
\usepackage{tcolorbox}
%\usepackage[T1]{fontenc}
%\usepackage{bookman}
\usepackage{subfigure}
\usepackage{hyperref}
\usepackage{listings}
\usepackage{animate}
\usepackage[absolute,overlay]{textpos}
\usepackage{graphicx}
\usepackage{tikz}
\usepackage[americaninductors,europeanresistors]{circuitikz}
\usepackage{tikz}
\usepackage{fancybox}     %% 定义zhushadow时用到
\usepackage{pifont} %ding用到
\newsavebox{\mysaveboxOne}  %%为了在only中使用lstlisting
\newsavebox{\mysaveboxTwo}
\newsavebox{\mysaveboxThree}
\newsavebox{\mysaveboxFour}
\newsavebox{\mysaveboxFive}
\newsavebox{\mysaveboxSix}
\newsavebox{\mysaveboxSeven}
\newcommand\zhushadow[2][purple]{\hskip5pt\shadowbox{\color{#1}\small\kai #2\vspace{3mm}}}

%%------------------------ThemeColorFont------------------------
%% Presentation Themes
% \usetheme[<options>]{<name list>}
%\usetheme{Madrid}
\usetheme{Berkeley}
%% Inner Themes双精度计算
% \useinnertheme[<options>]{<name>}
%% Outer Themes
% \useoutertheme[<options>]{<name>}
%\useoutertheme{miniframes} 
%% Color Themes 
%\usecolortheme[<options>]{<name list>}
%% Font Themes
\usefonttheme{serif}
\setbeamertemplate{background canvas}[vertical shading][bottom=white,top=structure.fg!7] %%背景色, 上25%的蓝, 过渡到下白.
\setbeamertemplate{theorems}[numbered]
\setbeamertemplate{navigation symbols}{}   %% 去掉页面下方默认的导航条.
\usepackage{styles/zhfontcfg}
%\setsansfont[Mapping=tex-text]{文泉驿正黑}  %% 需要fontspec宏包
     %如果装了Adobe Acrobat,可在font.conf中配置Adobe字体的路径以使用其中文字体
     %也可直接使用系统中的中文字体如SimSun,SimHei,微软雅黑 等
     %原来beamer用的字体是sans family;注意Mapping的大小写,不能写错
     %设置字体时也可以直接用字体名,以下三种方式等同:
     %\setromanfont[BoldFont={黑体}]{宋体}
     %\setromanfont[BoldFont={SimHei}]{SimSun}
     %\setromanfont[BoldFont={"[simhei.ttf]"}]{"[simsun.ttc]"}
%%------------------------MISC------------------------
\graphicspath{{figures/}}         %% 图片路径. 本文的图片都放在这个文件夹里了.
%%------------------------listing------------------------
%\lstset{language=[LaTeX]TeX,Python}
%%------------------------正文------------------------
\begin{document}
\XeTeXlinebreaklocale "zh"         % 表示用中文的断行
\XeTeXlinebreakskip = 0pt plus 1pt % 多一点调整的空间
%%----------------------------------------------------------
%% This is only inserted into the PDF information catalog. Can be left
%% out.
%%%
%% Delete this, if you do not want the table of contents to pop up at
%% the beginning of each subsection:
%\AtBeginSection[]{                              % 在每个Section前都会加入的Frame
%  \frame<handout:0>{
%    \frametitle{Contents}\small
%    \tableofcontents[current,currentsubsection]
%  }
%}
%
%\AtBeginSubsection[]                            % 在每个子段落之前
%{
%  \frame<handout:0>                             % handout:0 表示只在手稿中出现
%  {
%    \frametitle{Contents}\small
%    \tableofcontents[current,currentsubsection] % 显示在目录中加亮的当前章节
%  }
%}

\setbeamertemplate{caption}{\raggedright\insertcaption\par}

%%----------------------------------------------------------
\logo{\includegraphics[scale=0.07]{ouc-logo.jpg}}
\title{Saliency Detection}
%\subtitle{Bottom-Up Saliency Detection Model Based on Human Visual Sensitivity and Amplitude Spectrum}
\subtitle{显著性检测}
\author[]{\textcolor{olive}{Yafei Zhu\quad Lin Chang}}
\institute[CVBIOUC]
{
\small\textcolor{violet}{CVBIOUC\\
%Ocean University of China\\
\url{http://vision.ouc.edu.cn/~zhenghaiyong}}
}
\date[March 25, 2016]{March 25, 2016}
%\titlegraphic{
%\includegraphics[height=1.0cm]{ouc-logo.jpg}}
\frame{ \titlepage }
%%----------------------------------------------------------
%\section*{Contents}
\frame{\frametitle{Contents}\tableofcontents}
%%----------------------------------------------------------
\def\hilite<#1>{\temporal<#1>{\color{blue!15}}{\color{black}}{\color{black}}}
\newcommand{\shadow}[2][purple]{\hskip5pt\shadowbox{\color{#1}\small \kai #2\vspace{3mm}}}
\newcommand{\colorrbox}[2][purple]{\doublebox{\color{#1}\small \kai#2}}

%============================================================================

\section{Background}

%==========================================================================


\begin{frame}[fragile]
\frametitle{What?}
\begin{columns}
\begin{column}{\leftmargini}
\end{column}
\begin{column}{0.5\linewidth}
Saliency(显著)

\begin{itemize}
\item 明显、引入注目
\end{itemize}
\end{column}
\pause
\begin{column}{0.16\linewidth}
\huge $\Longleftarrow$ 
\end{column}
\pause
\begin{column}{0.3\linewidth}
 \centering{\color{blue}{Visual Attention}}
\end{column}
\end{columns}\vspace{1ex}


\begin{figure}
  \centering
\includegraphics[width=2in]{tianxianbaobao.jpg}
  \caption{What did you see?}
 \label{tianxianbaobao}
  \end{figure}
\end{frame}

\begin{frame}
\frametitle{What?}
  A rich stream of visual data enters our eyes every second\footnote{K. Koch \textit{et al.}, ``How much the eye tells the brain'', Current Biology, 2006. }.\newline
  
  \pause
  \centering{\Huge \color{blue}{$10^8 \sim 10^9 bits$}}
\end{frame}
  
  
\begin{frame}
  \frametitle{What?}
  \begin{center}
  \includegraphics[width=0.85\linewidth]{tuxiang.png}
  \end{center}
  
  \pause
  \centering{\huge $500 \times 375 \times 3 \times 8 = {\color{blue}4500000} bits$}
\end{frame}

  
\begin{frame}
\frametitle{What?}
  \begin{center}
  \includegraphics[width=1\linewidth]{people}
  \end{center}
\end{frame}


\begin{frame}
\frametitle{Vision as data reduction}
\begin{columns}
\begin{column}{\leftmargini}
\end{column}
\begin{column}{0.55\linewidth}
\begin{itemize}
\item Raw feed from camera/eyes:
\begin{itemize}
\item[-] {\huge \color{blue}$10^{7-9}$} Bytes/s
\end{itemize}
\item Extraction of edges and salient features
\begin{itemize}
\item[-] {\huge \color{blue}$10^{3-4}$} Bytes/s
\end{itemize}
\item High-level interpretation of scene
\begin{itemize}
\item[-] {\huge \color{blue}$10^{1-2}$} Bytes/s
\end{itemize}
\end{itemize}
\end{column}

\begin{column}{0.45\linewidth}
\centering\includegraphics[width=1.5in]{friendshipLove}
\end{column}
\end{columns}\vspace{1ex}

\end{frame}


\begin{frame}
\frametitle{Visual Attention}
\only<1>{
  \begin{center}
 \includegraphics[width=0.8\linewidth]{apple1}  
  \newline
  \usebox{\mysaveboxOne}
  \end{center}}
  
  \only<2>{
  \begin{center}
  \includegraphics[width=0.65\linewidth]{orientation}
  \newline
  \usebox{\mysaveboxTwo}
  \end{center}}

 \only<3>{
  \begin{center}
  \includegraphics[width=0.8\linewidth]{complex}
  \newline
  \usebox{\mysaveboxThree}
  \end{center}}
\end{frame}


\begin{frame}
\frametitle{Bottom-up vs Top-down}
\centering\includegraphics[width=9cm]{BottomupTopdown.png}
\end{frame}


\begin{frame}
\frametitle{Visual Attention}
\centering\includegraphics[width=10cm]{fields.png}
\end{frame}


\begin{frame}
\frametitle{Theory}
\begin{itemize}
\item ``{\color{blue}\emph{Feature Integration Theory}}'', Treisman \& Gelade, 1980.
\item ``{\color{blue}\emph{Guided Search Theory}}'', Wolfe \& Cave, 1989.
\item ``{\color{blue}\emph{Integrated Competition Theory}}'', Desimone \& Duncan, 1995.
\end{itemize}
\end{frame}


\begin{frame}
\frametitle{Framework}
\begin{itemize}
\item Koch and Ullman\footnote{C. Koch and S. Ullman, ``Shifts in selective visual attention: towards the underlying neural circuitry'', Human Neurobiology, 1985.} proposed a feed-forward model to combine features and introduced the concept of a {\color{blue}\emph{saliency map}} which is a topographic map that represents conspicuousness of scene locations.
\item Zoom Lens Model\footnote{Eriksen, C. W. and J. D. St. James, ``Visual attention within and around the field of focal attention: a zoom lens model'', Perception and Psychophysics, 1986.}
\end{itemize}
\end{frame}


\begin{frame}
\frametitle{Basic Structure of Computational Models}
\centering\includegraphics[width=10cm]{BasicStructure.jpg}
\end{frame}


\section{Development}

\begin{frame}
\frametitle{Two Waves\footnote{Ali Borji \textit{et al.}, ``Salient object detection: a survey'', 2014.}}
\centering\includegraphics[width=10cm]{wave}
\end{frame}


\begin{frame}
\frametitle{Fixation Prediction}
Fixation prediction models are constructed originally to understand human visual attention and eye movement prediction.

\centering\includegraphics[width=8cm]{fixationPrediction.png}
\end{frame}


\begin{frame}
\frametitle{Fixation Prediction:example}
\only<1>{
  \begin{center}
 \includegraphics[width=0.9\linewidth]{movement.png}  
  \newline
  \usebox{\mysaveboxOne}
  \end{center}}
  
  \only<2>{
  \begin{center}
  \includegraphics[width=0.9\linewidth]{movement1.png}
  \newline
  \usebox{\mysaveboxTwo}
  \end{center}}

 \only<3>{
  \begin{center}
  \includegraphics[width=0.9\linewidth]{movement2.png}
  \newline
  \usebox{\mysaveboxThree}
  \end{center}}
  
  \only<4>{
  \begin{center}
  \includegraphics[width=0.9\linewidth]{movement3.png}
  \newline
  \usebox{\mysaveboxFour}
  \end{center}}
  
  \only<5>{
  \begin{center}
  \includegraphics[width=0.9\linewidth]{movement4.png}
  \newline
  \usebox{\mysaveboxFive}
  \end{center}}
  
  \only<6>{
  \begin{center}
  \includegraphics[width=0.9\linewidth]{movement5.png}
  \newline
  \usebox{\mysaveboxSix}
  \end{center}}
\end{frame}


\begin{frame}
\frametitle{Representative Work}
\begin{columns}
\begin{column}{\leftmargini}
\end{column}
\begin{column}{0.6\linewidth}
\begin{itemize}
\item A model of saliency-based visual attention for rapid scene analysis. PAMI 1998, Itti \textit{et al.}.
\end{itemize}
\end{column}
\begin{column}{0.45\linewidth}
\centering\includegraphics[width=1in]{itti}
\end{column}
\end{columns}\vspace{1ex}

\begin{figure}[!ht]
  \begin{minipage}[t]{0.45\textwidth}
  \includegraphics[width=1.8in]{sign}
  \end{minipage}
  \begin{minipage}[t]{0.45\textwidth}
  \includegraphics[width=1.8in]{signSaliency}
  \end{minipage}
  \end{figure} 
\end{frame}


\begin{frame}
\frametitle{Architecture}
\begin{figure}
\includegraphics[width=10cm]{ITTImodel.png}
\end{figure}
\end{frame}

%
%\begin{frame}
%\frametitle{General Procedure of fixation prediction models}
%\centering\includegraphics[width=10cm]{fixationArchitecture}
%\end{frame}


%\begin{frame}
%\frametitle{How to evaluate fixation prediction models?}
%\centering\includegraphics[width=10cm]{evaluateFixationModel}
%\end{frame}


\begin{frame}
\frametitle{Salient Object Detection}
\centering\includegraphics[width=7cm]{FT.png}
\begin{itemize}
\item Learning to detect a salient object. CVPR 2007, Tie Liu \textit{et al.}.
\item Frequency-tuned salient region detection. CVPR 2009, Achanta \textit{et al.}.
\end{itemize}
\end{frame}


%\begin{frame}
%\frametitle{FT (\color{yellow} \textbf{Demo})}
%\centering\includegraphics[width=7cm]{FT.png}
%\begin{itemize}
%\item Learning to detect a salient object. CVPR 2007, Tie Liu et al.
%\item Frequency-tuned salient region detection. CVPR 2009, Achanta et al.
%\end{itemize}
%\end{frame}


\begin{frame}
\frametitle{State-of-the-art}
\centering\includegraphics[width=6cm]{latest}

  From left to right: PBS\footnote{Chuan Yang \textit{et al.}, ``Graph-regularized saliency detection with convex-hull-based center prior, Signal Processing Letters, 2013.}, DRFI\footnote{Huaizu Jiang \textit{et al.}, ``Salient object detection: a discriminative regional feature integration approach'', in CVPR, 2013.}, RBD\footnote{Wangjiang Zhu \textit{et al.}, ``Saliency optimization from robust background detection'', in CVPR, 2014.}, HDCT\footnote{Jiwhan Kim \textit{et al.}, ``Salient region detection via high-dimensional color transform'', in CVPR, 2014.}
\end{frame}


\begin{frame}
\frametitle{Why? }
The emergence of salient object detection models is driven by the requirement of saliency-based applications.
\begin{figure}[!ht]
  \begin{minipage}[t]{0.4\textwidth}
  \includegraphics[height=0.7in]{resizing}
  \caption{Content aware resizing}
  \end{minipage}
  \begin{minipage}[t]{0.4\textwidth}
  \includegraphics[height=0.7in]{manipulation}
  \caption{Object maniputation}
  \end{minipage}

\hspace{0.28in}
  \begin{minipage}[t]{0.4\textwidth}
  \includegraphics[height=0.7in]{sketch2photo}
  \caption{Image montage}
  \end{minipage}
  \begin{minipage}[t]{0.4\textwidth}
  \hspace{0.35in}
  \includegraphics[height=0.7in]{collage.jpg}
  \caption{Image collage}
  \end{minipage}
\end{figure} 
\end{frame}


\begin{frame}
\frametitle{What can saliency not do?}
\centering\includegraphics[width=9cm]{challenge.JPG}
\end{frame}


\begin{frame}
\frametitle{General Procedure}
\centering\includegraphics[width=4cm]{flowchart.png}
\end{frame}


\section{Features}

\begin{frame}
\frametitle{Features}
Salient object detection: A discriminative regional feature integration approach, CVPR 2013
\begin{itemize}
\item 26维颜色纹理等对比度特征

\includegraphics[width=8cm]{ContrastFeature.png}
\item 34维的区域特征

\includegraphics[width=8cm]{RegionProperty.png}
\end{itemize}
\end{frame}


\begin{frame}
\frametitle{Features}
\begin{itemize}
\item Pixel-based
\begin{itemize}
\item Contrast: $U_p(x) = \sum_{x'\in I\textbackslash \{ x\}}D(x', x)$
\item Property: Geometric features
\end{itemize}
\item Patch/Region/Superpixel-based
\begin{itemize}
\item {\color{magenta}Contrast}: $U_r(A_i) = \sum_{1\le k \le N, k \ne i}|\Lambda|D(\Lambda_k, \Lambda_i)$
\item {\color{magenta}Property}: Geometric features/Appearance features
\end{itemize}
\item Multi-scale based
\end{itemize}
\end{frame}


\begin{frame}
\frametitle{\ding{172} Contrast}
\begin{itemize}
\item Color
\item Texture
\end{itemize}
\end{frame}


\begin{frame}
\frametitle{Color}
\begin{itemize}
\item {\color{blue}\emph{Color}} is the visual perceptual property corresponding in humans to the categories called red, blue, yellow and others.
\item {\color{blue}\emph{Color Space}} is defined to identify colors numerically by their coordinates.
\end{itemize}
\end{frame}


\begin{frame}
\frametitle{Three Elements of Color}
\begin{itemize}
\item Hue: What we think as ``color''--yellow, orange, cyan and magenta are examples of different hues
\item Saturation: Saturation refers to the relative purity or the amount of white light mixed with a hue
\item Intensity
\end{itemize}
\centering\includegraphics[width=6cm]{HSIColorModel.jpg}
\end{frame}


\begin{frame}
\frametitle{Coding methods for humans}
\begin{itemize}
\item \textbf{RGB} is an additive system (add colors to black) used for displays.
\item \textbf{CMY} is a subtractive system for printing.
\item \textbf{HSI} is a good perceptual space for art, psychology, and recognition.
\item \textbf{YIQ} is used for TV and is good for compression. 
\end{itemize}
\end{frame}


\begin{frame}
\frametitle{Color Space}
\begin{itemize}
\item RGB
\item Lab
\item HSV
\end{itemize}
\end{frame}


\begin{frame}
\frametitle{Color Histogram\footnote{Huaizu Jiang \textit{et al.}, ``Salient object detection: a discriminative regional feature integration approach'', in CVPR, 2013.} }
\begin{itemize}
\item Quantize each color channel to have 16 different values
\item Compute matrix $Q\_rgb$
\item Superpixel segmentation
\item Compute color histogram of each superpixel
\item Compute color histogram contrast feature of each super pixel
\end{itemize}
\end{frame}

\begin{frame}
\frametitle{Color Histogram}
\begin{figure}
  \centering\includegraphics[width=1in]{1.jpg}
  \caption{Image}
 \label{1}
  \end{figure}

\begin{figure}
  \centering\includegraphics[width=3in]{RGBhist.jpg}
  \caption{Color histogram}
 \label{RGBhist}
  \end{figure}
\end{frame}


\begin{comment}
\begin{frame}
\frametitle{What is texture?}
{\color{blue}\emph{Texture}} is actually a very nebulous concept, often attributed to human perception, as either the feel or the appearance of (woven) fabric\footnote{Mark S.Nixon and Alberto S.Aguado, ``Feature Extraction \& Image Processing for Computer Vision'', Publishing House of Electronics Industry, 2013.}.
\begin{figure}[!ht]
  \begin{minipage}[t]{0.35\textwidth}
  \includegraphics[width=1.3in]{texture1.jpg}
  \end{minipage}
  \begin{minipage}[t]{0.35\textwidth}
  \includegraphics[width=1.3in]{texture2.jpg}
  \end{minipage}
  \begin{minipage}[t]{0.35\textwidth}
  \includegraphics[width=1.3in]{texture3.jpg}
  \end{minipage}
  \begin{minipage}[t]{0.35\textwidth}
  \includegraphics[width=1.3in]{texture4.jpg}
  \end{minipage}
  \end{figure} 
\end{frame}

%
%\begin{frame}
%\frametitle{Texture Descriptor}
%\begin{itemize}
%\item Structural approaches
%\item Statistical approaches
%\item Combination approaches
%\item Local binary patterns
%\end{itemize}
%\end{frame}


\begin{frame}
\frametitle{LBP (Local Binary Pattern)}
\centering\includegraphics[width=10cm]{LBP.png}
\end{frame}


\begin{frame}
\frametitle{LBP}
  \begin{figure}[!ht]
  \begin{minipage}[t]{0.46\textwidth}
  \includegraphics[width=1.8in]{example1.jpg}
  \end{minipage}
  \begin{minipage}[t]{0.46\textwidth}
  \includegraphics[width=1.8in]{LBP1.png}
  \end{minipage}
  \end{figure} 
\end{frame}


\begin{frame}
\frametitle{\ding{173} Regional property descriptor}
\begin{itemize}
\item Geometric feature descriptor 
\begin{itemize}
\item size
\item perimeter
\item position
\item area of the neighbor regions
\end{itemize}
\item Appearance feature descriptor
\begin{itemize}
\item the variances of the RGB values
\item the variances of the response of the LM filters
\end{itemize}
\end{itemize}
\end{frame}
\end{comment}

\section{Priors}

\begin{frame}
\frametitle{Priors}
\begin{enumerate}
\item Center Prior/Location Prior
\item Backgroundness Prior
\item Boundary Connectivity Prior
\item Color Prior
\item Objectness Prior
\item Smoothness Prior
\end{enumerate}
\end{frame}


\begin{frame}
\frametitle{\ding{172} Center Prior/Location Prior}
{\color{blue}Objects near the image center are more attractive to people}\footnote{Lin Zhang \textit{et al.}, ``SDSP: A novel saliency detection method by combining simple priors'', in ICIP, 2013.}{\color{blue}.}

\vspace{0.15in}

This prior can be simply and effectively modeled as a Gaussian map.

\vspace{0.15in}

\begin{columns}
\begin{column}{\leftmargini}
\end{column}
\hspace{-1in}
\begin{column}{0.1\linewidth}
\centering\includegraphics[width=3cm]{CenterPrior}
\end{column}
\begin{column}{0.7\linewidth}
\begin{align}
S_D(\textbf{x}) = exp\left(-\frac{||\textbf{x}-\textbf{c}||_2^2}{\sigma_D^2}\right)
\end{align}
\end{column}
\end{columns}\vspace{1ex}
\vspace{-0.4in}
\end{frame}


\begin{frame}
\frametitle{\ding{172} Center Prior/Location Prior}
{\color{blue}The salient object in an image is most probably placed near the center of the image}\footnote{Huaizu Jiang \textit{et al.}, "Automatic salient object segmentation based on context and shape prior", in BMVC, 2011}{\color{blue}.}

\vspace{0.2in}

Gaussian falloff weight:
\begin{align}
w_i^{(n)} = exp\left(-9\frac{(dx_i^{(n)})^2}{w^2}-9\frac{(dy_i^{(n)})^2}{h^2}\right)
\end{align}

\centering\includegraphics[width=3cm]{CBcolorPrior.png}
\end{frame}


\begin{frame}
\frametitle{\ding{172} Center Prior/Location Prior}
Assigning higher saliency to the image elements near the image center becomes invalid when the objects are placed far off the image center\footnote{Chuan Yang \textit{et al.}, ``Graph-regularized saliency detection with convex-hull-based center prior'', Signal Processing Letters, 2013.}.
\begin{columns}
\begin{column}{\leftmargini}
\end{column}
\begin{column}{0.5\linewidth}
\begin{itemize}
\item Compute a convex hull enclosing interesting points to estimate the location of salient region.
\item Use the centroid of the convex hull as the center to get the convex-hull-based center prior map.
\end{itemize}
\end{column}
\begin{column}{0.5\linewidth}
\centering\includegraphics[width=4cm]{convexHull}
\end{column}
\end{columns}\vspace{1ex}
\end{frame}

\begin{frame}
\frametitle{Center Prior/Location Prior}
\begin{figure}
  \centering
\includegraphics[width=4in]{center_prior.png}
  \caption{Center prior}
 \label{center prior}
  \end{figure}
\end{frame}


\begin{frame}
\frametitle{\ding{173} Backgroundness Prior}
{\color{blue}\emph{Backgroundness prior}} is more general than center prior because salient objects can be placed off the center, but they seldom touch the image boundary\footnote{Yichen Wei \textit{et al.}, ``Geodesic saliency using background priors'', in ECCV, 2012.}.
\begin{itemize}
\item Assuming that a narrow border of the image is background region, regional saliency can be computed as the contrast versus ``background''.
\end{itemize}
\vspace{0.3in}
\end{frame}

\begin{frame}
\frametitle{\ding{173} Backgroundness Prior}
\centering\includegraphics[width=0.7\textwidth]{GS.png}

\begin{align}
saliency(P) = \min_{P_1=P, P_2, \ldots, P_n=B}\sum_{i=1}^{n-1}weight(P_i, P_{i+1}), \\
s.t.(P_i, P_{i+1})\in \varepsilon \nonumber
\end{align}
\end{frame}

\begin{frame}
\frametitle{Backgroundness Prior}
\begin{figure}
  \centering
\includegraphics[width=3.5in]{backgroundness.png}
  \caption{Backgroundness prior}
 \label{center prior}
  \end{figure}
\end{frame}

\begin{frame}
\frametitle{\ding{174} Boundary Connectivity Prior}
{\color{blue}Object regions are much less connected to image boundaries than background ones}\footnote{Wangjiang Zhu \textit{et al.}, ``Saliency optimization from robust background detection'', in CVPR, 2014.}{\color{blue}.}

\centering\includegraphics[width=4cm]{wCtr.png}

\emph{Boundary connectivity} is defined to quantify how heavily a region R is connected to the image boundaries.
\begin{align}
BndCon(R) = \frac{|\{p|p\in R, p\in Bnd\}|}{\sqrt{|\{p|p\in R\}|}}
\end{align}
\end{frame}


\begin{frame}
\frametitle{\ding{175} Color Prior}
{\color{blue}Warm colors, such as red and yellow, are more pronounced to the human visual system than cold colors, such as green and blue}\footnote{Lin Zhang \textit{et al.}, ``SDSP: A novel saliency detection method by combining simple priors'', in ICIP, 2013.}{\color{blue}.}
\begin{align}
f_{an}(\textbf{x}) & =\frac{f_a(x)-min(a)}{max(a)-min(a)}, f_{bn}(\textbf{x}) = \frac{f_b(x)-min(b)}{max(b)-min(b)}\\
S_c(\textbf{x}) & = 1-exp\left(-\frac{f_{an}^2(\textbf{x})+f_{bn}^2(\textbf{x})}{\sigma_c^2}\right)
\end{align}
\begin{itemize}
\item $a^*$-channel represents green-red information
\item $b^*$-channel represents blue-yellow information
\end{itemize}
\end{frame}

\begin{frame}
\frametitle{Color Prior}
\begin{figure}
  \centering
\includegraphics[width=3.5in]{color_prior.png}
  \caption{Color prior}
 \label{center prior}
  \end{figure}
\end{frame}


\begin{frame}
\frametitle{\ding{176} Objectness Prior}
{\color{blue}\emph{Objectness}} is defined as the probability of there being a complete object in a local window centered on each pixel\footnote{Peng Jiang \textit{et al.}, ``Salient region detection by ufo: uniqueness, focusness and objectness'', in ICCV, 2013.}.
\begin{itemize}
\item Randomly sample $N$ windows over the image 
\item Assign each window $w$ a probability score $P(w)$ to indicate its objectness
\item Sum all the probability scores in windows that contains pixel $x$
\end{itemize}
\begin{align}
O_p(x) = \sum_{w \in W \text{ }and \text{ } x \in w} P(W_x)
\end{align}
\end{frame}


\begin{frame}
\frametitle{\ding{177} Smoothness Prior}
{\color{blue}\emph{Smoothness}} constraint is often encoded by adding a pair-wise potential to the energy function which encourages neighboring pixels in the image to take the same label\footnote{Chuan Yang \textit{et al.}, ``Graph-regularized saliency detection with convex-hull-based center prior'', Signal Processing Letters, 2013.}.
\begin{align}
w_{ij} & = exp\left(-\frac{||c_i-c_j||}{2\sigma_w^2}\right)\\
E(S) & = \sum_{i}(S(i)-S_{in}(i))^2+\lambda \sum_{i, j}w_{ij}(S(i)-S(j))^2
\end{align}
\end{frame}


\begin{frame}
\frametitle{\ding{177} Smoothness Prior}
\centering\includegraphics[width=10cm]{smoothness}
\end{frame}




\section{Aggregation}

\begin{frame}
\frametitle{\ding{172} Non-linear\footnote{Peng Jiang \textit{et al.}, ``Salient region detection by ufo: uniqueness, focusness and objectness'', in ICCV, 2013. }}
Combining the focusness, the objectness and the uniqueness maps:
\begin{align}
S = exp( F + U) \times O
\end{align}
\end{frame}

\begin{frame}
\frametitle{\ding{173} Energy minization\footnote{Wangjiang Zhu \textit{et al.}, ``Saliency optimization from robust background detection'', in CVPR, 2014.}}
The ideal output of salient object detection is a clean binary object/background segmentation.

\begin{align}
\underbrace{\sum_{i=1}^{N}w_i^{bg}s_i^2}_{background}+\underbrace{\sum_{i=1}^{N}w_i^{fg}(s_i-1)^2}_{foreground}+\underbrace{\sum_{i, j}w_{ij}(s_i-s_j)^2}_{smoothness}
\end{align}
\end{frame}


\begin{frame}
\frametitle{\ding{174} Machine Learning\footnote{Huaizu Jiang \textit{et al.}, ``Salient object detection: a discriminative regional feature integration approach'', in CVPR, 2013.}}
Training examples: 
\begin{itemize}
\item a set of confident regions $R = \{ R_1, R_2, \ldots, R_Q\}$
\item the responding saliency scores $A = \{a_1, a_2, \ldots, a_Q\}$
\end{itemize}
\end{frame}

\begin{frame}
\frametitle{Example: SDSP\footnote{Lin Zhang \textit{et al.}, ``SDSP: A novel saliency detection method by combining simple priors'', in ICIP, 2013.} }

\centering\includegraphics[width=5.7cm]{SDSP}
\end{frame}

\begin{frame}
\frametitle{Example: SDSP}
\begin{figure}
  \centering
\includegraphics[width=3in]{3.jpg}
  \caption{Image}
 \label{sdsp3}
  \end{figure}
\end{frame}

\begin{frame}
\frametitle{Example: SDSP}
\begin{figure}
  \centering
\includegraphics[width=4in]{sdsp4.png}
  \caption{SDSP}
 \label{sdsp3}
  \end{figure}
\end{frame}



\begin{frame}
\frametitle{Example: SDSP}
\begin{figure}
  \centering
\includegraphics[width=3in]{sdsp1.jpg}
  \caption{Image}
 \label{sdsp1}
  \end{figure}
\end{frame}

\begin{frame}
\frametitle{Example: SDSP}
\begin{figure}
  \centering
\includegraphics[width=4in]{sdsp2.png}
  \caption{SDSP}
 \label{sdsp2}
  \end{figure}
\end{frame}



\begin{frame}
  \vspace{2cm}
  \centering
  \zhushadow{\color{blue}\Huge{Thanks}}\\
  \vspace{1.5cm}
  \begin{flushright}
  \emph{\href{zhuyafei4520@163.com}{Yafei~Zhu\quad Lin~Chang}}\\
  \href{http://www.ouc.edu.cn}{Ocean University of China}\\
  \emph{2016.03}
  \end{flushright}  
\end{frame}



%===============================================================================


\end{document}
